%
\chapter{Creating video-summaries}\label{chp:video_summaries}
%
In the previous chapter we computed sets of labels for each frame in the dataset. All consecutive frames, which share a label is considered a segment. Each label says something about the content in a segment. By grouping the footage by label we get an overview of the different type of content we have. We now look at using these groups and labels for creating the final video summaries of the events depicted in the footage. 
%
\section{Method}
%
In order to utilise the labeling of the different footage we need be able to define a query (or \textit{recipe}), which describes the struture of the video summary we would like to create. Section \ref{sec:recipe} describes how we do this.\\
Then, in section \ref{sec:candidates}, we describe how we perform a preliminary search for decent candidates for each element in the recipe. In section \ref{sec:segment_score} we describe how we search through the videos and find the best possible segment. We define a metric, a \textit{fulfilment} score, which is used to to rank the segment, and finally, in section \ref{sec:choosing_segment}, choose which ones to use.
%
\subsection{Recipe}\label{sec:recipe}
%
The recipe for a video summary consist of one or more ingredients in a specific order, much like a cooking recipe. Each ingredient specifies a segment request, ie. an ingredient expresses desires on what labels a segment should contain. A segment is a sub-part of a video with a set of labels assigned to it. An ingredient is described by:
%
\begin{enumerate}
	\item requested labels - labels present in a segment has a positive impact on the segment score
	\item required labels - all these labels should be present in the segment
	\item forbidden labels - none of these label should be present in the segment
	\item min. span - segments should be no shorter than this
	\item max. span - segments should be no longer than this
	\item $\alpha$  - a ratio defining how the label- and time span- requirements are weighted against each other
\end{enumerate}
%
For each ingridient, we want to find a segment, which fits the ingredient definition as much as possible. The sequence of all such segments are used as the final video summary.
%
\subsection{Candidates}\label{sec:candidates}
%
In order to avoid having to analyse the entire dataset, we start by generating a smaller set of likely candidates. These candidates are found by looking for the overlap of requested labels, $l$ in a video, $v$. Let $O_i(v,l)$ define the number of areas in $v$ where $i$ labels overlap. We then define a candidate score for each video as:
%
\[
C(v, l) = \sum_{i=1}^{|l|} O_i(v,l)^{i},
\]
%
Overlap of labels are rewarded exponentially as a function of the amount of labels in the overlap, in order to reward videos with footage that fit our request well.\\
The list of videos are sorted by their candidate score and the most promising (the ones with the highest score) are passed on to the segment scoring phase (section \ref{sec:segment_score}), which performs a more thorough analysis.
%
\subsection{Segment Score}\label{sec:segment_score}
%
We perform a frame-by-frame analysis on each candidate video in order to determine what sub-part of it best fit the requirements established by the ingredient. This analysis is performed on all parts of the video, which has not already been used for the video summary we are in the process of generating. The most fitting sub-part is passed along to the final segment chooser (described in section \ref{sec:choosing_segment}).\\
The analysis is done on two aspects of \textit{fulfilment}. First we look at how well a sub-part fulfills the label requirement established by the ingredient. Then we look at its time span requirement. These two levels of fulfilment (defined as ratios) are then finally weighed together.
%
\subsubsection{Label fulfilment}
%
For each video in the candidate group we analyse each frame in regards to the labels present (or not present) in it. From this we generate a graph representing how well each frame throughout the video fulfills the label requirements we have for the segment to be chosen. Let $L_{x}$ be the set of requested labels, $L_{y}$ be the set required labels, and $L_{z}$ be the set of forbidden labels. Also let $f_{l}$ and $f_{x}$ be the total set of labels, and the set of requested labels present in frame $f$, respectively. The label requirement fulfilment-ratio for each frame is then defined as:\\
%
\begin{equation}
r(f) = \frac{|f_{x}|}{|L_{x}|} \cdot Y(f) \cdot Z(f),
\end{equation} 
%
where
%
\begin{equation}
Y(f) =
\begin{cases}
1 & \text{if} f_{l} \cap L_{y} = L_{y}\\
0 &  \text{otherwise}
\end{cases},
\end{equation} 
%
and
%
\begin{equation}
Z(f) =
\begin{cases}
1 & \text{if} f_{l} \cap L_{z} = \emptyset\\
0 &  \text{otherwise}
\end{cases}.
\end{equation} 
%
Effectively this means that a frame will receive a fulfilment-ratio of 0 if it does not contain a required label or if it contains a forbidden one, and the ratio of the number of requested labels it contains, otherwise.\\
%
With this fulfillment ratio across the entire span of the video we now have a tool of measure to identify the parts of it, that best fit the label requirements.
%
\subsubsection{Time span fulfilment}
%
The other aspect used to determine the score of a sub-part of a video is how well it fits within the time span we are looking to fill. A minimum- and maximum- length defines the time span, that we want the final segment to cover. Let $T_{min}$ and $T_{max}$ be the the minimum- and maximum- length, respectively, and let $l$ be the length of the sub-part in frames, where $T_min <= l <= T_max$. The time span fulfilment ratio is then defined as:\\
%
\begin{equation}
\tau(l) =
\begin{cases}
1 & \text{, if } T_{max} = T_{min}\\
\frac{l-T_{min}}{T_{max}-T_{min}} &  \text{, otherwise}
\end{cases}
\end{equation}\label{equ:fulfilment}
%
A sub-part of minimum length will thus have time span fulfilment ratio of zero, while one of maximum length will have a ratio of one. 
%
\subsubsection{The final score}
%
The score for a sub-part is based on how well it fulfils the label- as well as the time span-requirement. A ratio, $\alpha$, determines how the two are weighed against each other. Let $R$ be a set of the label fulfilment ratios for all frames in the video, as defined in \ref{equ:fulfilment}. Also, for each possible sub-part (that does not break the minimum- and maximum- lengths defined in the time span requirement), let $v$ be the frame number where the sub-part begins, and $l$ be the length of it in frames. The score for a sub-part is then defined as:\\
%
\begin{equation}
S(v,l) =(1-\alpha) \cdot \sum_{i=v}^{v+l} \frac{R(i)}{l} + \alpha \cdot \tau(l)
\end{equation}\label{equ:segment_score}
%
The score is thus determined by the average label fulfilment rate in the sub-part, weighed against the time span it covers.\\
We calculate a segment score for all valid sub-parts and pass the one with the highest score on to the segment chooser (section \ref{sec:choosing_segment}). If two sub-parts have share same score their lengths is used as a tiebreaker (the longest one is selected). The value of $\alpha$ can be used to configure the algorithm to produce segments, which either fulfill the label- or the time span- requirement very well. In most cases one would probably aim for some reasonable trade-off.
%
%
\subsection{Choosing a Segment}\label{sec:choosing_segment}
%
We now select one of the generated segments to be used in the final video summary. Let $\{v_1,v_2,\dots,v_n\}$ be the set of all segments generated from the candidate video, and let $w_i$ be the score of a segment $v_i$. For each segment, the probability of selection is defined as:
%
\[
P(v_i) = \frac{w_i^2}{W},
\]
%
where the sum of all scores, $W$, is defined as:
%
\[
W = \sum_{i=0}^{|w|} w_i^2
\]
%
To increase the probability of selection of higher scoring segments, we square all scores.
%