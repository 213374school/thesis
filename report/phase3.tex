%
\section{Creating video-summaries}
%

%
\subsection{Segment Database}
%
The is simply a list of segments, where each segment consist of a number of values. The start and end frame for which a given label covers, and the respective video.
%
\subsection{Candidate Factory}
%
% candidate factory: get candidates with requested labels then compute overlapping segments and finally sort segments by number of overlapping labels ties resolved by segment frame-length and yet again ties resolved by ytid (to group them, but really not needed)
% overlapping segments has a small required segment length (6 frames)
% top candidates are resolved and their ytid is extracted and fed to segment score algorithm. segment is then selected with a propability as a function of their score: s^2/S, S=sum(scores^2) where scores = scores > 0

%
\subsection{Segment Score}
%
% Remember to mention that segments chosen in part 3 are marked as used and can't be chosen again in that video.

%
\subsection{Choosing a Segment}
%

%
\subsection{Recipe}
%
% recipies - simple approach. no immediate feedback. kinda skipped lit. study on this one. universal recipe. 2 permutations on each dataset (different alpha_span value), and added some required labels 
% recipe structure: list of ingredients (labels, min/max span, interval, span alpha, required/forbidden labels)

%
\subsubsection{Random}
%

%
\subsubsection{Designer}
%