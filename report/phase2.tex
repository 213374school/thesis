%
\section{Grouping and labeling}
%
\subsection{Literature Study}
%
% need to describe Haar cascade classifiers (performance and an outline of how they work)

%
\subsection{Method}
%
% how we implemented the Haar cascade classifier

%
\subsection{Detecting interesting regions in video-clips}
% kim: hører til i litt. study
Detecting suitable regions in the video-clips is not enough. The content in the regions also has to be interesting. Hanjalic, A. \cite{citeulike:405480} describes a way to identify such regions by measuring the level of excitement in a sports video-clip based on a manually selected set of features. These features can be both visuals (like the movement in an image or the change of camera positions) or audial (namely the energy contained in the audio track).% WHETHER SUCH AN APPROACH IS SUITABLE TO US IS YET TO BE DETERMINED BASED ON THE CONTENT IN OUR FUTURE DATA SET.