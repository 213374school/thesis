% This file is part of matrix2latex.

% matrix2latex is free software: you can redistribute it and/or modify
% it under the terms of the GNU General Public License as published by
% the Free Software Foundation, either version 3 of the License, or
% (at your option) any later version.

% matrix2latex is distributed in the hope that it will be useful,
% but WITHOUT ANY WARRANTY; without even the implied warranty of
% MERCHANTABILITY or FITNESS FOR A PARTICULAR PURPOSE.  See the
% GNU General Public License for more details.

% You should have received a copy of the GNU General Public License
% along with matrix2latex. If not, see <http://www.gnu.org/licenses/>.

\documentclass[12pt, a4paper, reqno]{amsart}
%\documentclass[a4paper, english]{IEEEtran}
\usepackage[english]{babel}
\usepackage[utf8]{inputenc}

%\usepackage[cmex10]{amsmath}    % Recommended by IEEEtran
%\interdisplaylinepenalty=2500   % Recommended by IEEEtran, long math
\usepackage{amsmath}
\usepackage{amsfonts, amssymb, amsthm}
\allowdisplaybreaks[1] % displaybreak in math is allowed but avoided

\usepackage[disable]{epstopdf}  %interferes with latexmk epstopdf option.
%\epstopdfsetup{suffix={}}

\usepackage{sagetex} % allows the use of sage/python code to be executed

\usepackage{graphicx}           % include figures
\usepackage{hyperref}           % links
\usepackage[amssymb]{SIunits}   % provides SI suffixes
\usepackage{units}              % provides \unit[1]{\centi\meter}
\usepackage{booktabs}           % Fancy tables
\usepackage{color}              % Color, must be on if inkscape figures are used

%\usepackage{beramono}
\usepackage{listings}           % include code
\lstset{language = python,
  basicstyle= \small\bfseries,
  stringstyle=\ttfamily,
  commentstyle=\textit,
  tabsize=2,
  showspaces = false,
  showtabs = false,
  showstringspaces = false,
  mathescape = false}

% Useful macros
\providecommand{\fixme}[1]{\textbf{{\textsc{FIXME:} #1 }}}

% Math, recognized by evaluateLaTeX.py and/or python
\DeclareMathOperator{\exptext}{exp}
\renewcommand{\exp}[1]{\ensuremath{e^{#1}}}
\providecommand{\abs}[1]{\ensuremath{\left|#1\right|}}
\providecommand{\eqdef}{\ensuremath{\equiv}}
\providecommand{\e}[1]{\ensuremath{\times 10^{#1}}} % scientific notation
\renewcommand{\*}{\ensuremath{\cdot}}
\renewcommand{\t}[1]{\ensuremath{{\text{#1}}}} %shorthand for \text
% parenthesis
\providecommand{\paran}[1]{\ensuremath{\left( #1 \right)}}
\providecommand{\paranb}[1]{\ensuremath{\left[ #1 \right]}}

% Enumerate:
% \renewcommand{\labelenumi}{\arabic{enumi}}
% \renewcommand{\labelenumii}{\alph{enumii})}
% \renewcommand{\labelenumiii}{\roman{enumiii}}

% Typesett Roman numerals
\makeatletter
\newcommand{\rmnum}[1]{\romannumeral #1}
\newcommand{\Rmnum}[1]{\expandafter\@slowromancap\romannumeral #1@}
\makeatother

% Math stuff
\newcommand{\field}[1]{\mathbb{#1}}
\newcommand{\R}{\field{R}}
\newcommand{\C}{\field{C}}
\renewcommand{\P}{\field{P}}
\newcommand{\inv}[1]{#1^{-1}}

% Physics stuff
\newcommand{\emf}{\mathcal{E}}
\renewcommand{\epsilon}{\varepsilon}

% Figure width for one or two columns
\makeatletter
\newlength \figwidth
\if@twocolumn
  \setlength \figwidth {\columnwidth}
\else
  \setlength \figwidth {\textwidth}
\fi
\makeatother

\author{Øystein Bjørndal}
